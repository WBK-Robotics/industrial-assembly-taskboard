%% Generated by Sphinx.
\def\sphinxdocclass{report}
\documentclass[letterpaper,10pt,english]{sphinxmanual}
\ifdefined\pdfpxdimen
   \let\sphinxpxdimen\pdfpxdimen\else\newdimen\sphinxpxdimen
\fi \sphinxpxdimen=.75bp\relax
\ifdefined\pdfimageresolution
    \pdfimageresolution= \numexpr \dimexpr1in\relax/\sphinxpxdimen\relax
\fi
%% let collapsible pdf bookmarks panel have high depth per default
\PassOptionsToPackage{bookmarksdepth=5}{hyperref}

\PassOptionsToPackage{booktabs}{sphinx}
\PassOptionsToPackage{colorrows}{sphinx}

\PassOptionsToPackage{warn}{textcomp}
\usepackage[utf8]{inputenc}
\ifdefined\DeclareUnicodeCharacter
% support both utf8 and utf8x syntaxes
  \ifdefined\DeclareUnicodeCharacterAsOptional
    \def\sphinxDUC#1{\DeclareUnicodeCharacter{"#1}}
  \else
    \let\sphinxDUC\DeclareUnicodeCharacter
  \fi
  \sphinxDUC{00A0}{\nobreakspace}
  \sphinxDUC{2500}{\sphinxunichar{2500}}
  \sphinxDUC{2502}{\sphinxunichar{2502}}
  \sphinxDUC{2514}{\sphinxunichar{2514}}
  \sphinxDUC{251C}{\sphinxunichar{251C}}
  \sphinxDUC{2572}{\textbackslash}
\fi
\usepackage{cmap}
\usepackage[T1]{fontenc}
\usepackage{amsmath,amssymb,amstext}
\usepackage{babel}



\usepackage{tgtermes}
\usepackage{tgheros}
\renewcommand{\ttdefault}{txtt}



\usepackage[Bjarne]{fncychap}
\usepackage{sphinx}

\fvset{fontsize=auto}
\usepackage{geometry}


% Include hyperref last.
\usepackage{hyperref}
% Fix anchor placement for figures with captions.
\usepackage{hypcap}% it must be loaded after hyperref.
% Set up styles of URL: it should be placed after hyperref.
\urlstyle{same}


\usepackage{sphinxmessages}




\title{Industrial Assembly Taskboard}
\date{Jun 06, 2025}
\release{1.0.0}
\author{Jan Baumgärtner, Laurin Kreft}
\newcommand{\sphinxlogo}{\vbox{}}
\renewcommand{\releasename}{Release}
\makeindex
\begin{document}

\ifdefined\shorthandoff
  \ifnum\catcode`\=\string=\active\shorthandoff{=}\fi
  \ifnum\catcode`\"=\active\shorthandoff{"}\fi
\fi

\pagestyle{empty}
\sphinxmaketitle
\pagestyle{plain}
\sphinxtableofcontents
\pagestyle{normal}
\phantomsection\label{\detokenize{index::doc}}
\noindent{\hspace*{\fill}\sphinxincludegraphics[width=1000\sphinxpxdimen]{{logo_with_text}.png}\hspace*{\fill}}

\sphinxAtStartPar
Welcome to the documentation for the \sphinxstyleemphasis{Industrial Assembly Task Board},
a modular robotic benchmark developed by the wbk Institute of Production Science at Karlsruhe Institute of Technology.
This resource provides comprehensive instructions for constructing the task board hardware and detailed guidelines for evaluating robotic systems
on a variety of benchmark tasks.
Whether you are referencing this material online or as a compiled document,
you will find all the necessary information to build, configure, and utilize the task board for research, development, or benchmarking purposes.



\noindent{\hspace*{\fill}\sphinxincludegraphics[width=1000\sphinxpxdimen]{{sample_taskboards}.png}\hspace*{\fill}}

\sphinxAtStartPar
Robotic assembly is far more complex than the basic “peg\sphinxhyphen{}in\sphinxhyphen{}hole” tasks often used in benchmarks like the NIST taskboard.
While such benchmarks are valuable, they assume that the only thing more complex than a simple peg\sphinxhyphen{}in\sphinxhyphen{}hole task is the geometry of the peg and hole,
or the manipulation of cables and belts.
In reality, industrial assembly tasks involve a wide range of challenges, including precise tolerances, compliant assembly,
and the need for advanced planning and adaptability since different parts of an assembly may interact in unexpected ways.

\sphinxAtStartPar
The goal of this task board is to provide a set of tasks that showcases these additional complexities encountered in industrial practice.
These tasks are designed to challenge robotic agents with scenarios that require advanced planning, precision, and adaptability.
The modular design of the board allows users to mix and match tasks depending on the specific complexities they wish to address.

\sphinxAtStartPar
We provide all 3D\sphinxhyphen{}printable files in this repository for anyone to build their own task boards: \sphinxurl{https://github.com/WBK-Robotics/industrial-assembly-taskboard}

\begin{sphinxadmonition}{note}{Note:}
\sphinxAtStartPar
This task board is not intended to be a complete benchmark for all aspects of robotic assembly.
Instead, it serves as a starting point for exploring the complexities of industrial assembly tasks.
We encourage users to extend the task board with their own modules and tasks that reflect real\sphinxhyphen{}world challenges.
If you have ideas for new tasks, please contribute by opening a Pull Request with your files and task descriptions.
\end{sphinxadmonition}


\chapter{What you’ll find here}
\label{\detokenize{index:what-youll-find-here}}
\sphinxAtStartPar
This documentation is split into two main sections:
\begin{itemize}
\item {} 
\sphinxAtStartPar
\sphinxstylestrong{Assembly Guide} \textendash{} step‑by‑step hardware build instructions (3D‑printing,

\end{itemize}
\begin{quote}

\sphinxAtStartPar
purchased fasteners, mounting sequence).
\end{quote}
\begin{itemize}
\item {} 
\sphinxAtStartPar
\sphinxstylestrong{Robot Task Guide} \textendash{} formal task definitions (motion sequence, tolerances and

\end{itemize}
\begin{quote}

\sphinxAtStartPar
success criteria) for benchmarking your robot or algorithm on each module.
\end{quote}

\begin{sphinxadmonition}{note}{Note:}
\sphinxAtStartPar
If you are new to the benchmark, follow the quick‑start steps below. Power
users can jump directly to the Assembly or Robot Task guides via the sidebar.
\end{sphinxadmonition}


\section{Quick‑Start}
\label{\detokenize{index:quickstart}}
\sphinxAtStartPar
1. Read the {\hyperref[\detokenize{wbk_challenge_overview::doc}]{\sphinxcrossref{\DUrole{doc}{Assembly Guide}}}} to build the base taskboard and all
desired task modules.

\sphinxAtStartPar
2. Choose a module and open its page in the {\hyperref[\detokenize{wbk_challenge_robot_tasks_overview::doc}]{\sphinxcrossref{\DUrole{doc}{Task Descriptions}}}}
to understand the robot’s motions and evaluation metrics.
\begin{enumerate}
\sphinxsetlistlabels{\arabic}{enumi}{enumii}{}{.}%
\setcounter{enumi}{2}
\item {} 
\sphinxAtStartPar
Implement your robot program, run experiments, and compare results!

\end{enumerate}


\chapter{Contents}
\label{\detokenize{index:contents}}
\sphinxstepscope


\section{Assembly Guide}
\label{\detokenize{wbk_challenge_overview:assembly-guide}}\label{\detokenize{wbk_challenge_overview::doc}}
\sphinxAtStartPar
The \sphinxstyleemphasis{industrial assembly challenge} taskboard is a modular benchmark developed at the wbk Institute of Production Science to evaluate and compare robotic assembly strategies.
Each module focuses on a distinct industrial skill—from simple peg‑in‑hole insertions to assembling parts with compliant elastic deformation.

\sphinxAtStartPar
This online documentation collects every set of step‑by‑step instructions you need to \sphinxstylestrong{3D‑print, source hardware, and assemble} the taskboard and all of its interchangeable sub‑tasks.


\subsection{How to use this guide}
\label{\detokenize{wbk_challenge_overview:how-to-use-this-guide}}\begin{enumerate}
\sphinxsetlistlabels{\arabic}{enumi}{enumii}{}{.}%
\item {} 
\sphinxAtStartPar
\sphinxstylestrong{Print or purchase the required components} listed for each module.

\item {} 
\sphinxAtStartPar
\sphinxstylestrong{Follow the illustrated assembly sequence} described on the corresponding page.

\item {} 
\sphinxAtStartPar
\sphinxstylestrong{Mount the finished module} onto the taskboard base or use it stand\sphinxhyphen{}alone for experiments.

\end{enumerate}

\begin{sphinxadmonition}{tip}{Tip:}
\sphinxAtStartPar
Start with \DUrole{xref,std,std-doc}{assembly\_instructions\_taskboard} to build the base board, then add tasks from the submodules as desired.
\end{sphinxadmonition}


\subsection{Build Instructions}
\label{\detokenize{wbk_challenge_overview:build-instructions}}
\sphinxstepscope


\subsubsection{Assembling Instructions: Task board}
\label{\detokenize{0-Assembly_Instructions_Taskboard:assembling-instructions-task-board}}\label{\detokenize{0-Assembly_Instructions_Taskboard::doc}}

\paragraph{Step 1: Prepare the Components}
\label{\detokenize{0-Assembly_Instructions_Taskboard:step-1-prepare-the-components}}

\paragraph{Step 2: Position the Task Board}
\label{\detokenize{0-Assembly_Instructions_Taskboard:step-2-position-the-task-board}}
\sphinxAtStartPar
Place the 3D\sphinxhyphen{}printed task board upside down.
Ensure that the four rectangular holes are visible.

\noindent{\hspace*{\fill}\sphinxincludegraphics[width=400\sphinxpxdimen]{{taskboard_foot_placement}.png}\hspace*{\fill}}


\paragraph{Step 3: Placing the Feet}
\label{\detokenize{0-Assembly_Instructions_Taskboard:step-3-placing-the-feet}}
\sphinxAtStartPar
Insert each task board foot into one of the rectangular holes.
Make sure the small bar for fixation is pointing away from the center.

\noindent{\hspace*{\fill}\sphinxincludegraphics[width=400\sphinxpxdimen]{{taskboard_foot_fixation}.png}\hspace*{\fill}}


\paragraph{Step 4: Secure the Feet with Screws}
\label{\detokenize{0-Assembly_Instructions_Taskboard:step-4-secure-the-feet-with-screws}}
\sphinxAtStartPar
Place an M8 nut in the hexagonal hole on the bottom of the foot.
Lift the board and insert a 30mm M8 screw through the hole on the opposite side.
Begin tightening the screw, ensuring it properly connects with the nut.
Repeat this process for all four feet.

\noindent{\hspace*{\fill}\sphinxincludegraphics[width=400\sphinxpxdimen]{{taskboard_final}.png}\hspace*{\fill}}

\sphinxstepscope


\subsubsection{Assembly Instructions: Peg\sphinxhyphen{}in\sphinxhyphen{}Hole}
\label{\detokenize{1-Assembly-Instructions-Peg-in-Hole:assembly-instructions-peg-in-hole}}\label{\detokenize{1-Assembly-Instructions-Peg-in-Hole::doc}}
\sphinxAtStartPar
This section provides step\sphinxhyphen{}by\sphinxhyphen{}step instructions for assembling the Peg\sphinxhyphen{}in\sphinxhyphen{}Hole task module of the Industrial Assembly Task Board. The module consists of a round peg and a splined shaft peg that need to be inserted into their respective holes.
In contrast to other boards such as the NIST peg\sphinxhyphen{}in\sphinxhyphen{}hole tasks the focus here is on tight tolerances and precise insertion of the pegs into the holes,
rather than on complex peg and hole geometries.

\noindent{\hspace*{\fill}\sphinxincludegraphics[width=400\sphinxpxdimen]{{peg_in_hole_taskboard}.png}\hspace*{\fill}}


\paragraph{Task 1: Round Peg in Hole}
\label{\detokenize{1-Assembly-Instructions-Peg-in-Hole:task-1-round-peg-in-hole}}

\subparagraph{Step 1: Prepare the Components}
\label{\detokenize{1-Assembly-Instructions-Peg-in-Hole:step-1-prepare-the-components}}

\subparagraph{Step 2: Inserting the Plain Bearing into the Hole component}
\label{\detokenize{1-Assembly-Instructions-Peg-in-Hole:step-2-inserting-the-plain-bearing-into-the-hole-component}}
\sphinxAtStartPar
Push the Plain Bearing from the top with force to insert it into the Hole Round component.

\noindent{\hspace*{\fill}\sphinxincludegraphics[width=200\sphinxpxdimen]{{bearing_in_hole}.png}\hspace*{\fill}}


\subparagraph{Step 3: Manual Fixation of the Assembly on the Task\sphinxhyphen{}Board}
\label{\detokenize{1-Assembly-Instructions-Peg-in-Hole:step-3-manual-fixation-of-the-assembly-on-the-task-board}}
\sphinxAtStartPar
Align the two side holes of the Assembly part with two suitable holes on the Taskboard. Then push two 30mm M8 screws through the aligned holes and secure them from the bottom of the Taskboard with two M8 nuts.

\noindent{\hspace*{\fill}\sphinxincludegraphics[width=400\sphinxpxdimen]{{round_peg_in_hole_assembly_on_taskboard}.png}\hspace*{\fill}}


\subparagraph{Step 4: Placing the Round Shaft}
\label{\detokenize{1-Assembly-Instructions-Peg-in-Hole:step-4-placing-the-round-shaft}}
\sphinxAtStartPar
Place the Shaft standing on a flat surface next to the task board so that the robot can grasp it.


\paragraph{Task 2: Splined Shaft Peg in Hole}
\label{\detokenize{1-Assembly-Instructions-Peg-in-Hole:task-2-splined-shaft-peg-in-hole}}

\subparagraph{Step 1: Prepare the Components}
\label{\detokenize{1-Assembly-Instructions-Peg-in-Hole:id1}}

\subparagraph{Step 2: Inserting the Plain Bearing into the Hole component}
\label{\detokenize{1-Assembly-Instructions-Peg-in-Hole:id2}}
\sphinxAtStartPar
Push the Splined Shaft Sleeve from the top with force to insert it into the Hole Splined Shaft component as shown in Task 1.


\subparagraph{Step 3: Manual Fixation of the Assembly on the Task\sphinxhyphen{}Board}
\label{\detokenize{1-Assembly-Instructions-Peg-in-Hole:id3}}
\sphinxAtStartPar
Align the two side holes of the Assembly part with two suitable holes on the Taskboard.
Then push two 30mm M8 screws through the aligned holes and secure them from the bottom of the Taskboard with M8 Nuts.


\subparagraph{Step 4: Placing the Splined Shaft}
\label{\detokenize{1-Assembly-Instructions-Peg-in-Hole:step-4-placing-the-splined-shaft}}
\sphinxAtStartPar
Place the Splined Shaft on a flat surface next to the Taskboard, so that the robot can grasp it.


\paragraph{Task 3: BNC Connector}
\label{\detokenize{1-Assembly-Instructions-Peg-in-Hole:task-3-bnc-connector}}

\subparagraph{Step 1: Prepare the Components}
\label{\detokenize{1-Assembly-Instructions-Peg-in-Hole:id4}}

\subparagraph{Step 2: Inserting the Female BNC\sphinxhyphen{}Connector into the Mounting Component}
\label{\detokenize{1-Assembly-Instructions-Peg-in-Hole:step-2-inserting-the-female-bnc-connector-into-the-mounting-component}}
\sphinxAtStartPar
Push the Female BNC\sphinxhyphen{}Connector from the top with force to insert it vertically into the Hole on the Mounting BNC Connector component.


\subparagraph{Step 3: Manual Fixation of the Assembly on the Task\sphinxhyphen{}Board}
\label{\detokenize{1-Assembly-Instructions-Peg-in-Hole:id5}}
\sphinxAtStartPar
Align the two side holes of the Assembly part with two suitable holes on the Taskboard.
Then push two 30mm M8 screws through the aligned holes and secure them from the bottom of the Taskboard with M8 nuts.


\subparagraph{Step 4: Placing the Splined Shaft}
\label{\detokenize{1-Assembly-Instructions-Peg-in-Hole:id6}}
\sphinxAtStartPar
Place the Male BNC\sphinxhyphen{}Connector on a flat surface next to the Taskboard, so that the robot can grasp it.

\sphinxstepscope


\subsubsection{Assembly Instructions: Pick and Place}
\label{\detokenize{2-Assembly-Instructions-Pick-and-Place:assembly-instructions-pick-and-place}}\label{\detokenize{2-Assembly-Instructions-Pick-and-Place::doc}}
\sphinxAtStartPar
This section describes the assembly instructions for the Pick and Place task module of the industrial assembly challenge taskboard.
The module consists of three tasks: Housing Parts, Feather Key, and Shim Ring placement.

\noindent{\hspace*{\fill}\sphinxincludegraphics[width=400\sphinxpxdimen]{{pick_and_place_taskboard}.png}\hspace*{\fill}}


\paragraph{Task 1: Housing Parts}
\label{\detokenize{2-Assembly-Instructions-Pick-and-Place:task-1-housing-parts}}

\subparagraph{Step 1: Prepare the Components}
\label{\detokenize{2-Assembly-Instructions-Pick-and-Place:step-1-prepare-the-components}}

\subparagraph{Step 2: Manual Fixation of the Housing Part1 on the Taskboard}
\label{\detokenize{2-Assembly-Instructions-Pick-and-Place:step-2-manual-fixation-of-the-housing-part1-on-the-taskboard}}
\sphinxAtStartPar
Align the two side holes of the Housing Part 1 component with any two suitable holes on the Taskboard. Then push two 30mm M8 screws through the aligned holes and secure them from the bottom of the Taskboard with M8 nuts.

\noindent{\hspace*{\fill}\sphinxincludegraphics[width=400\sphinxpxdimen]{{housing_part1_placement}.png}\hspace*{\fill}}

\sphinxAtStartPar
Place the Housing Part 2 on a flat surface next to the Taskboard, so that the Robot can grasp it.


\paragraph{Task 2: Feather Key}
\label{\detokenize{2-Assembly-Instructions-Pick-and-Place:task-2-feather-key}}

\subparagraph{Step 1: Prepare the Components}
\label{\detokenize{2-Assembly-Instructions-Pick-and-Place:id1}}

\subparagraph{Step 2: Manual Fixation of the Feather Key Assembly on the Taskboard}
\label{\detokenize{2-Assembly-Instructions-Pick-and-Place:step-2-manual-fixation-of-the-feather-key-assembly-on-the-taskboard}}
\sphinxAtStartPar
Align the two side holes of the Feather Key Assembly component with any two suitable holes on the Taskboard. Then push two 30mm M8 screws through the aligned holes and secure them with M8 nuts.
Place the Feather Key on a flat surface next to the Taskboard.


\paragraph{Task 3: Shim Ring}
\label{\detokenize{2-Assembly-Instructions-Pick-and-Place:task-3-shim-ring}}

\subparagraph{Step 1: Prepare the Components}
\label{\detokenize{2-Assembly-Instructions-Pick-and-Place:id2}}

\subparagraph{Step 2: Manual Assembly of the Shim Ring Task}
\label{\detokenize{2-Assembly-Instructions-Pick-and-Place:step-2-manual-assembly-of-the-shim-ring-task}}
\sphinxAtStartPar
Align the two side holes of the Shim Ring Assembly component with any two suitable holes on the Taskboard. Then push two 30mm M8 screws through the aligned holes and secure them with M8 nuts.

\noindent{\hspace*{\fill}\sphinxincludegraphics[width=400\sphinxpxdimen]{{shim_ring_assembly_placement}.png}\hspace*{\fill}}

\sphinxAtStartPar
Place the Shim Ring Magazine next to the Taskboard and place a shim ring in each of the 5 magazine slits as shown in Figure 2.

\sphinxstepscope


\subsubsection{Assembly Instructions: Gear Assembly}
\label{\detokenize{3-Assembly-Instructions-Gear-Assembly:assembly-instructions-gear-assembly}}\label{\detokenize{3-Assembly-Instructions-Gear-Assembly::doc}}
\sphinxAtStartPar
This section provides step\sphinxhyphen{}by\sphinxhyphen{}step instructions for assembling the Gear Assembly task module of the Industrial Assembly Task Board.
The module consists of various gear configurations that can be assembled and tested.

\noindent{\hspace*{\fill}\sphinxincludegraphics[width=400\sphinxpxdimen]{{gear_assembly_taskboard}.png}\hspace*{\fill}}


\paragraph{Task 1: 2 Gears}
\label{\detokenize{3-Assembly-Instructions-Gear-Assembly:task-1-2-gears}}

\subparagraph{Step 1: Prepare the Components}
\label{\detokenize{3-Assembly-Instructions-Gear-Assembly:step-1-prepare-the-components}}

\subparagraph{Step 2: Manual Assembly of the 2 Gears Task}
\label{\detokenize{3-Assembly-Instructions-Gear-Assembly:step-2-manual-assembly-of-the-2-gears-task}}
\sphinxAtStartPar
Align the two holes on the side of the Gear Mounting part with two suitable holes on the Taskboard. Then push two 30mm M8 Screws through the aligned holes and secure them from the bottom with M8 Nuts.
Place the two Gear 1 parts next to the Taskboard.


\paragraph{Task 2: Helical Gear}
\label{\detokenize{3-Assembly-Instructions-Gear-Assembly:task-2-helical-gear}}

\subparagraph{Step 1: Prepare the Components}
\label{\detokenize{3-Assembly-Instructions-Gear-Assembly:id1}}

\subparagraph{Step 2: Manual Assembly of the helical Gear Task}
\label{\detokenize{3-Assembly-Instructions-Gear-Assembly:step-2-manual-assembly-of-the-helical-gear-task}}
\sphinxAtStartPar
Align the two side holes of the Helical Gear Assembly with two suitable holes on the Taskboard. Then insert two 30mm M8 Screws through the aligned holes and secure them from the bottom of the Taskboard with M8 Nuts.
Finally, place the Helical Gear next to the Taskboard.


\paragraph{Task 3: 3 Gears}
\label{\detokenize{3-Assembly-Instructions-Gear-Assembly:task-3-3-gears}}

\subparagraph{Step 1: Prepare the Components}
\label{\detokenize{3-Assembly-Instructions-Gear-Assembly:id2}}

\subparagraph{Step 2: Manual Assembly of the 3 Gears Task Assembly}
\label{\detokenize{3-Assembly-Instructions-Gear-Assembly:step-2-manual-assembly-of-the-3-gears-task-assembly}}
\sphinxAtStartPar
Place Mounting Part 2 facing down like shown in the picture.

\sphinxAtStartPar
Then place the two Shafts with Key parts into the two holes of Mounting Part 2.

\noindent{\hspace*{\fill}\sphinxincludegraphics[width=400\sphinxpxdimen]{{3_gears_mounting_part_2}.png}\hspace*{\fill}}

\sphinxAtStartPar
Next place the mounting part 1 on top of the assembled components, so that the two shafts stick out.

\noindent{\hspace*{\fill}\sphinxincludegraphics[width=400\sphinxpxdimen]{{3_gears_mounting_part_1}.png}\hspace*{\fill}}

\sphinxAtStartPar
Place the assembled parts on the Taskboard holding them together, so that the two holes on the side of the assembly group line up with two holes on the Taskboard. Push the two 40mm M8 Screws through the holes and fixate them from the bottom of the board using the M8 Nuts.
Place the 3D\sphinxhyphen{}printed Gears, two with groove and one without next to the Taskboard on a flat surface graspable for the robot.

\sphinxstepscope


\subsubsection{Assembly Instructions: Screws and Nuts}
\label{\detokenize{4-Assembly-Instructions-Screws-and-Nuts:assembly-instructions-screws-and-nuts}}\label{\detokenize{4-Assembly-Instructions-Screws-and-Nuts::doc}}
\sphinxAtStartPar
This section provides step\sphinxhyphen{}by\sphinxhyphen{}step assembly instructions for tasks involving screws and nuts.
Each task includes the preparation of components, manual assembly steps, and fixation on the taskboard.
Note that the assembly of this task requires a soldering iron to insert thread inserts into the 3D\sphinxhyphen{}printed parts.

\noindent\sphinxincludegraphics{{screws_and_nuts_taskboard}.png}

\sphinxAtStartPar
    :alt: Screws and Nuts Task Module
    :align: center
    :width: 400px


\paragraph{Task 1: Hexagon Socket Screw}
\label{\detokenize{4-Assembly-Instructions-Screws-and-Nuts:task-1-hexagon-socket-screw}}

\subparagraph{Step 1: Prepare the Components}
\label{\detokenize{4-Assembly-Instructions-Screws-and-Nuts:step-1-prepare-the-components}}
\sphinxAtStartPar
    :header\sphinxhyphen{}rows: 1
\begin{quote}
\begin{itemize}
\item {} \begin{itemize}
\item {} 
\sphinxAtStartPar
3D\sphinxhyphen{}printed parts:

\end{itemize}

\end{itemize}
\begin{itemize}
\item {} 
\sphinxAtStartPar
Purchased components:

\end{itemize}
\begin{itemize}
\item {} \begin{itemize}
\item {} 
\sphinxAtStartPar
Screw Assembly × 1 (Material: PETG)

\end{itemize}

\end{itemize}
\begin{itemize}
\item {} 
\sphinxAtStartPar
30mm M8 Screw × 2

\end{itemize}
\begin{itemize}
\item {} \begin{itemize}
\item {} 
\end{itemize}

\end{itemize}
\begin{itemize}
\item {} 
\sphinxAtStartPar
20mm M8 Screw × 1

\end{itemize}
\begin{itemize}
\item {} \begin{itemize}
\item {} 
\end{itemize}

\end{itemize}
\begin{itemize}
\item {} 
\sphinxAtStartPar
M8 Nut × 2

\end{itemize}
\begin{itemize}
\item {} \begin{itemize}
\item {} 
\end{itemize}

\end{itemize}
\begin{itemize}
\item {} 
\sphinxAtStartPar
M8 Thread Insert × 1

\end{itemize}
\end{quote}


\subparagraph{Step 2: Manual Assembly of the Hexagonal Socket Screw Task}
\label{\detokenize{4-Assembly-Instructions-Screws-and-Nuts:step-2-manual-assembly-of-the-hexagonal-socket-screw-task}}
\sphinxAtStartPar
Place the Thread Insert over the hole of the Screw Assembly part, ensuring that the smaller diameter is on the bottom side.
Proceed with the following steps:
\begin{enumerate}
\sphinxsetlistlabels{\arabic}{enumi}{enumii}{}{.}%
\item {} 
\sphinxAtStartPar
Heat the insert \textendash{} Use a soldering iron with an M8 tip that fits into the insert. Heat the insert until it reaches a temperature where it can melt the surrounding plastic.

\item {} 
\sphinxAtStartPar
Press the insert in \textendash{} Gently apply pressure to push the insert into the hole while the plastic softens and conforms around it.

\item {} 
\sphinxAtStartPar
Allow cooling \textendash{} Once in place, let the plastic cool and solidify, ensuring a strong mechanical bond.

\end{enumerate}

\noindent\sphinxincludegraphics{{screw_assembly_heat_insert}.png}

\sphinxAtStartPar
    :alt: Screw Assembly
    :align: center
    :width: 200px


\subparagraph{Step 3: Manual Fixation of the Assembly on the Taskboard}
\label{\detokenize{4-Assembly-Instructions-Screws-and-Nuts:step-3-manual-fixation-of-the-assembly-on-the-taskboard}}
\sphinxAtStartPar
Align the two side holes of the Assembly component with two corresponding holes on the Taskboard of your choice. Insert two 30mm M8 screws through the aligned holes and secure them from the bottom of the Board with M8 nuts.
Additionally, place the 20mm M8 screw on a horizontal surface next to the board, ensuring it is positioned so that the robot can grasp it.


\paragraph{Task 2: Nut Assembly with Difficult Accessibility}
\label{\detokenize{4-Assembly-Instructions-Screws-and-Nuts:task-2-nut-assembly-with-difficult-accessibility}}

\subparagraph{Step 1: Prepare the Components}
\label{\detokenize{4-Assembly-Instructions-Screws-and-Nuts:id1}}
\sphinxAtStartPar
    :header\sphinxhyphen{}rows: 1
\begin{quote}
\begin{itemize}
\item {} \begin{itemize}
\item {} 
\sphinxAtStartPar
3D\sphinxhyphen{}printed parts:

\end{itemize}

\end{itemize}
\begin{itemize}
\item {} 
\sphinxAtStartPar
Purchased components:

\end{itemize}
\begin{itemize}
\item {} \begin{itemize}
\item {} 
\sphinxAtStartPar
Nut Assembly × 1 (Material: PETG)

\end{itemize}

\end{itemize}
\begin{itemize}
\item {} 
\sphinxAtStartPar
M8 Nut × 3

\end{itemize}
\begin{itemize}
\item {} \begin{itemize}
\item {} 
\end{itemize}

\end{itemize}
\begin{itemize}
\item {} 
\sphinxAtStartPar
30mm M8 Screw × 3

\end{itemize}
\begin{itemize}
\item {} \begin{itemize}
\item {} 
\end{itemize}

\end{itemize}
\begin{itemize}
\item {} 
\sphinxAtStartPar
M8 Thread Inserts × 1

\end{itemize}
\end{quote}


\subparagraph{Step 2: Manual Assembly of the Nut Assembly Task}
\label{\detokenize{4-Assembly-Instructions-Screws-and-Nuts:step-2-manual-assembly-of-the-nut-assembly-task}}
\sphinxAtStartPar
Place the Thread Insert over the central hole on the top of the Nut Assembly part, ensuring that the smaller diameter is on the bottom side.
Proceed with the following steps:
\begin{enumerate}
\sphinxsetlistlabels{\arabic}{enumi}{enumii}{}{.}%
\item {} 
\sphinxAtStartPar
Heat the insert \textendash{} Use a soldering iron with an M8 tip that fits into the insert. Heat the insert until it reaches a temperature where it can melt the surrounding plastic.

\item {} 
\sphinxAtStartPar
Press the insert in \textendash{} Gently apply pressure to push the insert into the hole while the plastic softens and conforms around it.

\item {} 
\sphinxAtStartPar
Allow cooling \textendash{} Once in place, let the plastic cool and solidify, ensuring a strong mechanical bond.

\end{enumerate}

\sphinxAtStartPar
Next, screw a 30mm M8 screw into the thread insert.

\noindent\sphinxincludegraphics{{nut_assembly_heat_insert}.png}

\sphinxAtStartPar
    :alt: Nut Assembly
    :align: center
    :width: 200px


\subparagraph{Step 3: Manual Fixation of the Assembly on the Taskboard}
\label{\detokenize{4-Assembly-Instructions-Screws-and-Nuts:id2}}
\sphinxAtStartPar
Align the two side holes of the Nut Assembly component with two suitable holes on the Taskboard. To ensure the robot has sufficient clearance for the Nut Assembly Task, the component should be mounted in a way that leaves the open side unobstructed. Mounting it on the side of the Taskboard ensures that the accessibility is not more difficult than intended.
Insert two 30mm M8 screws through the aligned holes and secure them from the bottom of the Board with M8 nuts.
Additionally, place the M8 screw on a horizontal surface next to the board, ensuring it is positioned for the robot to grasp.


\paragraph{Task 3: Multiple Screws with the Risk of Tilting}
\label{\detokenize{4-Assembly-Instructions-Screws-and-Nuts:task-3-multiple-screws-with-the-risk-of-tilting}}

\subparagraph{Step 1: Prepare the Components}
\label{\detokenize{4-Assembly-Instructions-Screws-and-Nuts:id3}}
\sphinxAtStartPar
    :header\sphinxhyphen{}rows: 1
\begin{quote}
\begin{itemize}
\item {} \begin{itemize}
\item {} 
\sphinxAtStartPar
3D\sphinxhyphen{}printed parts:

\end{itemize}

\end{itemize}
\begin{itemize}
\item {} 
\sphinxAtStartPar
Purchased components:

\end{itemize}
\begin{itemize}
\item {} \begin{itemize}
\item {} 
\sphinxAtStartPar
Screws Tilting Part1 × 1 (Material: PETG)

\end{itemize}

\end{itemize}
\begin{itemize}
\item {} 
\sphinxAtStartPar
M8 Nut × 2

\end{itemize}
\begin{itemize}
\item {} \begin{itemize}
\item {} 
\sphinxAtStartPar
Screws Tilting Part2 × 1 (Material: PETG)

\end{itemize}

\end{itemize}
\begin{itemize}
\item {} 
\sphinxAtStartPar
30mm M8 Screw × 2

\end{itemize}
\begin{itemize}
\item {} \begin{itemize}
\item {} 
\end{itemize}

\end{itemize}
\begin{itemize}
\item {} 
\sphinxAtStartPar
20mm M8 Screw × 4

\end{itemize}
\begin{itemize}
\item {} \begin{itemize}
\item {} 
\end{itemize}

\end{itemize}
\begin{itemize}
\item {} 
\sphinxAtStartPar
M8 Thread Insert × 4

\end{itemize}
\end{quote}


\subparagraph{Step 2: Manual Assembly of the Hexagonal Socket Screw Task}
\label{\detokenize{4-Assembly-Instructions-Screws-and-Nuts:id4}}
\sphinxAtStartPar
Put the 4 thread inserts into the middle holes of the Screws\_tilting\_part1 part, ensuring that the smaller diameter is on the bottom side.
Proceed with the following steps for each of the four thread inserts:
\begin{enumerate}
\sphinxsetlistlabels{\arabic}{enumi}{enumii}{}{.}%
\item {} 
\sphinxAtStartPar
Heat the insert \textendash{} Use a soldering iron with an M8 tip that fits into the insert. Heat the insert until it reaches a temperature where it can melt the surrounding plastic.

\item {} 
\sphinxAtStartPar
Press the insert in \textendash{} Gently apply pressure to push the insert into the hole while the plastic softens and conforms around it.

\item {} 
\sphinxAtStartPar
Allow cooling \textendash{} Once in place, let the plastic cool and solidify, ensuring a strong mechanical bond.

\end{enumerate}

\sphinxAtStartPar
Then place the Screws Tilting Part 2 on top of Part 1 lining up with the holes, that have the thread inserts. Step 3: Manual Fixation of the Assembly on the Taskboard Align the two side holes of the Screws Tilting Part 1 with two suitable holes on the Taskboard. Insert two 30mm M8 screws through the aligned holes and secure them from the bottom of the Board with M8 nuts.
Additionally, place the four 20mm M8 screws on a horizontal surface next to the board, ensuring they are positioned for the robot to grasp.

\noindent\sphinxincludegraphics{{screws_tilting_assembly}.png}

\sphinxAtStartPar
    :alt: Screws Tilting Assembly
    :align: center
    :width: 400px

\sphinxstepscope


\subsubsection{Assembly Instructions: Elastic Deformation}
\label{\detokenize{5-Assembly-Instructions-Elastic-Deformation:assembly-instructions-elastic-deformation}}\label{\detokenize{5-Assembly-Instructions-Elastic-Deformation::doc}}
\sphinxAtStartPar
This section provides step\sphinxhyphen{}by\sphinxhyphen{}step instructions for assembling the Elastic Deformation task module of the Industrial Assembly Task Board.
The module consists of three tasks: Snap Hook, Spring, and Circlip.

\noindent{\hspace*{\fill}\sphinxincludegraphics[width=400\sphinxpxdimen]{{elastic_deformation_taskboard}.png}\hspace*{\fill}}


\paragraph{Task 1: Snap Hook}
\label{\detokenize{5-Assembly-Instructions-Elastic-Deformation:task-1-snap-hook}}

\subparagraph{Step 1: Prepare the Components}
\label{\detokenize{5-Assembly-Instructions-Elastic-Deformation:step-1-prepare-the-components}}

\subparagraph{Step 2: Manual fixation of the housing on the Task Board}
\label{\detokenize{5-Assembly-Instructions-Elastic-Deformation:step-2-manual-fixation-of-the-housing-on-the-task-board}}
\sphinxAtStartPar
Align the two holes on the side of the Snap Hook Assembly component with any two suitable holes on the Taskboard. Then push two 30mm M8 screws through the aligned holes and secure them from the bottom of the Taskboard with M8 nuts.

\noindent{\hspace*{\fill}\sphinxincludegraphics[width=400\sphinxpxdimen]{{snap_hook_placement}.png}\hspace*{\fill}}

\sphinxAtStartPar
Place the Snap Hook on a flat surface next to the Taskboard so that the Robot can grasp it.


\paragraph{Task 2: Spring}
\label{\detokenize{5-Assembly-Instructions-Elastic-Deformation:task-2-spring}}

\subparagraph{Step 1: Prepare the Components}
\label{\detokenize{5-Assembly-Instructions-Elastic-Deformation:id1}}

\subparagraph{Step 2: Manual fixation of the housing on the Task Board}
\label{\detokenize{5-Assembly-Instructions-Elastic-Deformation:id2}}
\sphinxAtStartPar
Align the two holes on the side of the Pressure Spring Assembly component with any two suitable holes on the Taskboard. Then, push two 30mm M8 Screws through the aligned holes and secure them from the bottom of the Taskboard with M8 Nuts.
Place the Spring on a flat surface next to the Taskboard to allow the robot to grasp it.


\paragraph{Task 3: Circlip}
\label{\detokenize{5-Assembly-Instructions-Elastic-Deformation:task-3-circlip}}

\subparagraph{Step 1: Prepare the Components}
\label{\detokenize{5-Assembly-Instructions-Elastic-Deformation:id3}}

\subparagraph{Step 2: Manual fixation of the housing on the Task Board}
\label{\detokenize{5-Assembly-Instructions-Elastic-Deformation:id4}}
\sphinxAtStartPar
Align the two holes on the side of the Circlip Assembly component with any two suitable holes on the Taskboard. Then push two 30mm M8 screws through the aligned holes and secure them from the bottom of the Taskboard with M8 nuts.
Place the circlip on a flat surface next to the Taskboard to allow the robot to grasp it.

\sphinxstepscope


\subsubsection{List of required Components}
\label{\detokenize{List-of-Required-Components:list-of-required-components}}\label{\detokenize{List-of-Required-Components::doc}}
\sphinxAtStartPar
The following is a list of non\sphinxhyphen{}3D\sphinxhyphen{}printed components needed to assemble the different tasks.
The links and Misumi as a suggested supplier are merely recommendations. Equivalent components, with the measurements given from different suppliers, work just as well.
Screws:


\begin{savenotes}\sphinxattablestart
\sphinxthistablewithglobalstyle
\centering
\begin{tabulary}{\linewidth}[t]{TTTT}
\sphinxtoprule
\sphinxstyletheadfamily 
\sphinxAtStartPar
Component
&\sphinxstyletheadfamily 
\sphinxAtStartPar
Specification
&\sphinxstyletheadfamily 
\sphinxAtStartPar
Quantity
&\sphinxstyletheadfamily 
\sphinxAtStartPar
Notes
\\
\sphinxmidrule
\sphinxtableatstartofbodyhook
\sphinxAtStartPar
Screw M8\sphinxhyphen{}30mm SCB8\sphinxhyphen{}30
&
\sphinxAtStartPar
30mm length
&
\sphinxAtStartPar
100 + 50
&\\
\sphinxhline
\sphinxAtStartPar
Screw M8\sphinxhyphen{}20mm SCB8\sphinxhyphen{}20
&
\sphinxAtStartPar
20mm length
&
\sphinxAtStartPar
5
&\\
\sphinxhline
\sphinxAtStartPar
Screw M8\sphinxhyphen{}40mm SCB8\sphinxhyphen{}40
&
\sphinxAtStartPar
40mm length
&
\sphinxAtStartPar
2
&\\
\sphinxhline
\sphinxAtStartPar
Thread insert M8 UD\sphinxhyphen{}48001
&
\sphinxAtStartPar
M8
&
\sphinxAtStartPar
6
&\\
\sphinxhline
\sphinxAtStartPar
Thread insert M6 UD\sphinxhyphen{}46095
&
\sphinxAtStartPar
M6
&
\sphinxAtStartPar
2
&\\
\sphinxhline
\sphinxAtStartPar
Plain bearing MPBZ20\sphinxhyphen{}50
&
\sphinxAtStartPar
ID: 20mm, OD: 28mm, L: 50mm
&
\sphinxAtStartPar
1
&
\sphinxAtStartPar
OD can be changed if CAD is modified
\\
\sphinxhline
\sphinxAtStartPar
Shaft PSFG20\sphinxhyphen{}100
&
\sphinxAtStartPar
D: 20mm (f8), L: 100mm
&
\sphinxAtStartPar
1
&\\
\sphinxhline
\sphinxAtStartPar
Splined shaft sleeve 64821500
&
\sphinxAtStartPar
N 15x1.25x10, 10 teeth, ID: 12.5mm, OD: 15mm, Sleeve OD: 38mm, L: 40mm
&
\sphinxAtStartPar
1
&
\sphinxAtStartPar
OD/Length can be modified in CAD
\\
\sphinxhline
\sphinxAtStartPar
Splined shaft 64811510
&
\sphinxAtStartPar
W 15x1.25x10, 10 teeth, ID: 12.1mm, OD: 14.75mm, L: 215mm (min 100mm)
&
\sphinxAtStartPar
1
&
\sphinxAtStartPar
Min. 60mm splined, min. 100mm shaft
\\
\sphinxhline
\sphinxAtStartPar
BNC\sphinxhyphen{}Connector female BNC9025\sphinxhyphen{}BJ
&&
\sphinxAtStartPar
1
&\\
\sphinxhline
\sphinxAtStartPar
BNC\sphinxhyphen{}Connector male BNC\sphinxhyphen{}LP\sphinxhyphen{}3DW
&&
\sphinxAtStartPar
1
&\\
\sphinxhline
\sphinxAtStartPar
Spring WF20\sphinxhyphen{}80
&
\sphinxAtStartPar
L: 80mm, Compressible: 44mm, D: 20mm
&
\sphinxAtStartPar
1
&\\
\sphinxhline
\sphinxAtStartPar
Circlip STWN30
&
\sphinxAtStartPar
Size: 30 (DIN 471)
&
\sphinxAtStartPar
1
&\\
\sphinxhline
\sphinxAtStartPar
Shim ring K1151.3042050
&
\sphinxAtStartPar
W: 0.5mm, ID: 30mm, OD: 42mm
&
\sphinxAtStartPar
5
&\\
\sphinxhline
\sphinxAtStartPar
Shaft PSFG30\sphinxhyphen{}100
&
\sphinxAtStartPar
L: 100mm, D: 30mm
&
\sphinxAtStartPar
1
&
\sphinxAtStartPar
Can be 3D\sphinxhyphen{}printed
\\
\sphinxhline
\sphinxAtStartPar
Shaft with bore FPSFJBB\sphinxhyphen{}D30\sphinxhyphen{}L120\sphinxhyphen{}M8\sphinxhyphen{}N10
&
\sphinxAtStartPar
L: 120mm, W: 30mm, Bore: 8mm/10mm
&
\sphinxAtStartPar
1
&
\sphinxAtStartPar
Can be 3D\sphinxhyphen{}printed, bore min. 8mm
\\
\sphinxbottomrule
\end{tabulary}
\sphinxtableafterendhook\par
\sphinxattableend\end{savenotes}

\sphinxstepscope


\section{Task Descriptions}
\label{\detokenize{wbk_challenge_robot_tasks_overview:task-descriptions}}\label{\detokenize{wbk_challenge_robot_tasks_overview::doc}}
\sphinxAtStartPar
This section describes all tasks of the industrial assembly task board.
Like the assembly instructions, the individual tasks are organized in modules but can be mixed and matched freely to create custom benchmarks.


\subsection{How to use this guide}
\label{\detokenize{wbk_challenge_robot_tasks_overview:how-to-use-this-guide}}\begin{enumerate}
\sphinxsetlistlabels{\arabic}{enumi}{enumii}{}{.}%
\item {} 
\sphinxAtStartPar
Pick the tasks you want to benchmark.

\item {} 
\sphinxAtStartPar
Read its \sphinxstylestrong{robotic assembly task} page to understand the required motions and tolerances.

\item {} 
\sphinxAtStartPar
Implement and test your robot program until the success criteria are met.

\end{enumerate}


\subsection{Assembly Task}
\label{\detokenize{wbk_challenge_robot_tasks_overview:assembly-task}}
\sphinxstepscope


\subsubsection{Peg in Hole Tasks}
\label{\detokenize{robotic_instructions_peg_in_hole:peg-in-hole-tasks}}\label{\detokenize{robotic_instructions_peg_in_hole::doc}}
\sphinxAtStartPar
While there are many different peg\sphinxhyphen{}in\sphinxhyphen{}hole benchmarks, these typically focus on varying the peg and hole geometries,
but often feature large tolerances (in many cases such as the NIST taskboard, to be able to allow for the comparison of a wider range of robots since many have difficulty with tight tolerances).
In contrast, the tasks in this section focus on tight tolerances and precise insertion of the pegs into the holes.

\sphinxAtStartPar
The final state of all three tasks is shown in the taskboard image below,
where the robot has successfully inserted the pegs into their respective holes.

\noindent{\hspace*{\fill}\sphinxincludegraphics[width=400\sphinxpxdimen]{{peg_in_hole_taskboard_finished}.png}\hspace*{\fill}}


\paragraph{Task 1: Round Peg}
\label{\detokenize{robotic_instructions_peg_in_hole:task-1-round-peg}}

\subparagraph{Description:}
\label{\detokenize{robotic_instructions_peg_in_hole:description}}
\sphinxAtStartPar
A round peg is positioned upright on a flat surface adjacent to the Taskboard.
The task requires the peg to be inserted into a designated round hole in a target component.


\subparagraph{Test Objective:}
\label{\detokenize{robotic_instructions_peg_in_hole:test-objective}}
\sphinxAtStartPar
This task evaluates the robot’s ability to handle cylindrical objects, perform precise alignment,
and execute controlled insertion motions typical in industrial assembly processes.


\subparagraph{Success Criteria:}
\label{\detokenize{robotic_instructions_peg_in_hole:success-criteria}}\begin{itemize}
\item {} 
\sphinxAtStartPar
The round peg is fully inserted into the designated hole.

\item {} 
\sphinxAtStartPar
No displacement of the peg occurs after the robot releases it.

\end{itemize}


\paragraph{Task 2: Splined Shaft Peg}
\label{\detokenize{robotic_instructions_peg_in_hole:task-2-splined-shaft-peg}}

\subparagraph{Description:}
\label{\detokenize{robotic_instructions_peg_in_hole:id1}}
\sphinxAtStartPar
A splined shaft peg is positioned upright on a flat surface next to the Taskboard.
The task requires the peg to be inserted into a matching splined hole in a target component.


\subparagraph{Test Objective:}
\label{\detokenize{robotic_instructions_peg_in_hole:id2}}
\sphinxAtStartPar
This task assesses the robot’s capability to recognize and align keyed or splined features, ensuring correct orientation and fit during insertion.


\subparagraph{Success Criteria:}
\label{\detokenize{robotic_instructions_peg_in_hole:id3}}\begin{itemize}
\item {} 
\sphinxAtStartPar
The splined shaft peg is fully inserted into the matching splined hole.

\item {} 
\sphinxAtStartPar
No displacement of the peg occurs after the robot releases it.

\end{itemize}


\paragraph{Task 3: BNC\sphinxhyphen{}Connector}
\label{\detokenize{robotic_instructions_peg_in_hole:task-3-bnc-connector}}

\subparagraph{Description:}
\label{\detokenize{robotic_instructions_peg_in_hole:id4}}
\sphinxAtStartPar
A male BNC connector is positioned on a flat surface next to the Taskboard.
The task requires the connector to be inserted and locked into a corresponding female socket on the Taskboard.


\subparagraph{Test Objective:}
\label{\detokenize{robotic_instructions_peg_in_hole:id5}}
\sphinxAtStartPar
This task tests the robot’s ability to perform insert\sphinxhyphen{}and\sphinxhyphen{}twist operations, requiring both translational and rotational alignment,
as well as engagement of locking mechanisms.


\subparagraph{Success Criteria:}
\label{\detokenize{robotic_instructions_peg_in_hole:id6}}\begin{itemize}
\item {} 
\sphinxAtStartPar
The BNC connector is inserted such that both lugs of the plug reach the end of the guide slots in the socket.

\item {} 
\sphinxAtStartPar
The connector is rotated to the locked position, as verified by a mechanical stop or electrical continuity check.

\end{itemize}

\sphinxstepscope


\subsubsection{Pick and Place Tasks}
\label{\detokenize{robotic_instructions_pick_and_place:pick-and-place-tasks}}\label{\detokenize{robotic_instructions_pick_and_place::doc}}
\sphinxAtStartPar
This section provides the task descriptions for the Pick and Place task module of the industrial assembly challenge taskboard.
The module consists of three tasks: Housing Parts, Feather Key, and Shim Ring placement.
The focus here is on adaptivity, the robot needs to sense the assembly to determine the correct placement and alignment of parts,
rather than reaching a fixed target pose like in the Peg\sphinxhyphen{}in\sphinxhyphen{}Hole task.

\sphinxAtStartPar
A sample taskboard with all three tasks finished is shown below,
where the robot has successfully completed all tasks by placing the parts in the correct positions and orientations.

\noindent{\hspace*{\fill}\sphinxincludegraphics[width=400\sphinxpxdimen]{{pick_and_place_taskboard_finished}.png}\hspace*{\fill}}


\paragraph{Task 1: Housing Parts}
\label{\detokenize{robotic_instructions_pick_and_place:task-1-housing-parts}}

\subparagraph{Description:}
\label{\detokenize{robotic_instructions_pick_and_place:description}}
\sphinxAtStartPar
In real\sphinxhyphen{}world assembly, parts often need to be aligned based on subtle visual cues or features that are difficult to detect,
rather than relying solely on fixed orientations or clearly visible holes as in simple “peg\sphinxhyphen{}in\sphinxhyphen{}hole” problems.
This task simulates such a scenario where the robot must align two housing parts based on matching features,
such as drill holes and flange surfaces, which may not be perfectly aligned or visible to the robot’s sensors.
A housing part is placed on a flat surface next to the Taskboard. Another housing part is already mounted on the Taskboard.
The robot must pick up the loose housing part and place it centrally on top of the mounted part, ensuring correct alignment.

\noindent{\hspace*{\fill}\sphinxincludegraphics[width=400\sphinxpxdimen]{{pick_and_place_taskboard_housing_parts}.png}\hspace*{\fill}}


\subparagraph{Test Objective:}
\label{\detokenize{robotic_instructions_pick_and_place:test-objective}}
\sphinxAtStartPar
This task evaluates the robot’s ability to grasp, transport, and precisely align components with matching features,
simulating assembly operations that require accurate positioning and orientation.


\subparagraph{Success Criteria:}
\label{\detokenize{robotic_instructions_pick_and_place:success-criteria}}\begin{itemize}
\item {} 
\sphinxAtStartPar
The housing part is placed on top of the mounted component with parallel alignment.

\item {} 
\sphinxAtStartPar
The distance between the two flange surfaces is exactly 2 mm.

\item {} 
\sphinxAtStartPar
The center axes of the aligned drill holes match, with a maximum rotational deviation of 1°.

\item {} 
\sphinxAtStartPar
The housing part remains stationary after placement.

\end{itemize}


\paragraph{Task 2: Feather Key}
\label{\detokenize{robotic_instructions_pick_and_place:task-2-feather-key}}

\subparagraph{Description:}
\label{\detokenize{robotic_instructions_pick_and_place:id1}}
\sphinxAtStartPar
A feather key is a small, rectangular metal component used to connect rotating machine elements, such as gears or pulleys, to a shaft,
ensuring torque transmission while allowing for easy assembly and disassembly.
Feather keys fit into matching grooves (keyways) on both the shaft and the mating part, providing a secure, non\sphinxhyphen{}permanent connection.

\sphinxAtStartPar
In this task, a feather key is placed on a flat surface next to the Taskboard. T
he robot must pick up the feather key and insert it vertically into the groove of a shaft mounted on the Taskboard.

\noindent{\hspace*{\fill}\sphinxincludegraphics[width=400\sphinxpxdimen]{{pick_and_place_taskboard_feather_key}.png}\hspace*{\fill}}


\subparagraph{Test Objective:}
\label{\detokenize{robotic_instructions_pick_and_place:id2}}
\sphinxAtStartPar
This task evaluates the robot’s ability to handle small parts,
perform precise insertions, and maintain control during placement.


\subparagraph{Success Criteria:}
\label{\detokenize{robotic_instructions_pick_and_place:id3}}\begin{itemize}
\item {} 
\sphinxAtStartPar
The feather key is fully inserted into the groove on the shaft.

\item {} 
\sphinxAtStartPar
The feather key remains stationary after the robot releases it.

\end{itemize}


\paragraph{Task 3: Shim Ring}
\label{\detokenize{robotic_instructions_pick_and_place:task-3-shim-ring}}

\subparagraph{Description:}
\label{\detokenize{robotic_instructions_pick_and_place:id4}}
\sphinxAtStartPar
Shim rings are thin, precisely manufactured rings used in industry to adjust the axial position or spacing between machine components,
ensuring correct alignment, preload, or clearance.
In industrial settings, the required number of shim rings can vary depending on component tolerances,
wear, or assembly variations. Therefore, the robot must autonomously determine how many shim rings to stack to reach the correct height—specifically,
up to but not covering the groove on the shaft.
This reflects real\sphinxhyphen{}world scenarios where automated systems
must adapt to part variability and ensure assemblies meet strict tolerance requirements without manual intervention.

\noindent{\hspace*{\fill}\sphinxincludegraphics[width=400\sphinxpxdimen]{{pick_and_place_taskboard_shim_rings}.png}\hspace*{\fill}}


\subparagraph{Test Objective:}
\label{\detokenize{robotic_instructions_pick_and_place:id5}}
\sphinxAtStartPar
This task evaluates the robot’s ability to not just perform repetitive pick\sphinxhyphen{}and\sphinxhyphen{}place operations,
but follow them to reach a specified target height as required in many industrial assembly tasks.
In this case it autonomously determine when to stop based on a visual reference.


\subparagraph{Success Criteria:}
\label{\detokenize{robotic_instructions_pick_and_place:id6}}\begin{itemize}
\item {} 
\sphinxAtStartPar
Four shim rings are stacked onto the shaft, reaching the groove without covering it.

\item {} 
\sphinxAtStartPar
The number of shim rings is determined autonomously by the robot.

\item {} 
\sphinxAtStartPar
All placed shim rings remain stationary after placement.

\end{itemize}

\sphinxAtStartPar
A sample rendering of this state is shown below,
where the robot has successfully placed four shim rings on the shaft, reaching the groove without covering it.

\noindent{\hspace*{\fill}\sphinxincludegraphics[width=400\sphinxpxdimen]{{pick_and_place_taskboard_shim_rings_finished}.png}\hspace*{\fill}}

\sphinxstepscope


\subsubsection{Gear Assembly Tasks}
\label{\detokenize{robotic_instructions_gear_tasks:gear-assembly-tasks}}\label{\detokenize{robotic_instructions_gear_tasks::doc}}
\sphinxAtStartPar
This section describes the tasks related to gear assembly on the industrial assembly task board.
These tasks require the robot to handle various gear configurations, ensuring proper alignment, meshing, and seating of the gears on their respective shafts.
The following image provides an overview of a task board with all three gear assembly tasks completed successfully.

\noindent{\hspace*{\fill}\sphinxincludegraphics[width=400\sphinxpxdimen]{{gear_assembly_taskboard_finished}.png}\hspace*{\fill}}


\paragraph{Task 1: Two Gears}
\label{\detokenize{robotic_instructions_gear_tasks:task-1-two-gears}}

\subparagraph{Description:}
\label{\detokenize{robotic_instructions_gear_tasks:description}}
\sphinxAtStartPar
Two spur gears must be mounted onto parallel shafts of an assembly component so that their teeth mesh correctly.
The gears are initially placed on a flat surface next to the Taskboard.
The robot must ensure that the gears are positioned such that their teeth engage without interference and that each gear is fully seated on its respective shaft.
This task is basically a 3D printable version of the gear assembly task from the NIST assembly challenge.

\noindent{\hspace*{\fill}\sphinxincludegraphics[width=400\sphinxpxdimen]{{gear_assembly_taskboard_two_gears}.png}\hspace*{\fill}}


\subparagraph{Test Objective:}
\label{\detokenize{robotic_instructions_gear_tasks:test-objective}}
\sphinxAtStartPar
This task evaluates the robot’s ability to recognize gear orientation, align mounting holes with the shafts, and ensure proper meshing between gears.


\subparagraph{Success Criteria:}
\label{\detokenize{robotic_instructions_gear_tasks:success-criteria}}\begin{itemize}
\item {} 
\sphinxAtStartPar
Both gears are mounted onto the shafts with their teeth properly engaged.

\item {} 
\sphinxAtStartPar
The gears are fully seated on the assembly component, with no visible gaps.

\item {} 
\sphinxAtStartPar
The teeth of the two gears mesh without interference and can rotate together.

\end{itemize}


\paragraph{Task 2: Gear with Helical Teeth}
\label{\detokenize{robotic_instructions_gear_tasks:task-2-gear-with-helical-teeth}}

\subparagraph{Description:}
\label{\detokenize{robotic_instructions_gear_tasks:id1}}
\sphinxAtStartPar
A lot of gears in the physical world have more complicated tooth profiles than just spur gears.
This means that they cannot be inserted in a simple peg\sphinxhyphen{}in\sphinxhyphen{}hole manner.
Instead the gear must be rotated as it is lowered onto the shaft to ensure that the teeth engage correctly.
To simulate this, a helical gear with helical teeth must be mounted onto a shaft of the assembly component.
The gear is placed on a flat surface next to the Taskboard.
Due to the helical design, the gear must be rotated as it is lowered onto the shaft to ensure the teeth engage correctly with an existing gear.

\noindent{\hspace*{\fill}\sphinxincludegraphics[width=400\sphinxpxdimen]{{gear_assembly_taskboard_helical_gear}.png}\hspace*{\fill}}


\subparagraph{Test Objective:}
\label{\detokenize{robotic_instructions_gear_tasks:id2}}
\sphinxAtStartPar
This task assesses the robot’s ability to handle components with non\sphinxhyphen{}standard tooth profiles,
requiring simultaneous rotation and insertion. The robot must ensure that the gear is fully seated and that the helical teeth are properly engaged.


\subparagraph{Success Criteria:}
\label{\detokenize{robotic_instructions_gear_tasks:id3}}\begin{itemize}
\item {} 
\sphinxAtStartPar
The helical gear is mounted onto the shaft with its teeth correctly engaged with the mating gear.

\item {} 
\sphinxAtStartPar
The gear is fully seated, with no gap between its underside and the assembly component.

\item {} 
\sphinxAtStartPar
The gear remains stationary after placement and does not move when released.

\end{itemize}


\paragraph{Task 3: Three Gears with Grooves}
\label{\detokenize{robotic_instructions_gear_tasks:task-3-three-gears-with-grooves}}

\subparagraph{Description:}
\label{\detokenize{robotic_instructions_gear_tasks:id4}}
\sphinxAtStartPar
Three gears—two with grooves and one without—must be mounted onto three shafts of the assembly component.
The two grooved gears are placed on the outer shafts and must be aligned so that their grooves match the keys on the shafts.
The third gear, which does not have a groove, is mounted on the middle shaft and must be positioned so that its teeth engage with the two outer gears.

\sphinxAtStartPar
In this task, the gear teeth are intentionally made larger than in the previous assemblies.
This means that the robot cannot simply place the gears onto the shafts in any order and expect the teeth to mesh.
Instead, the robot must carefully plan the sequence of assembly and the orientation of each shaft and gear.
If the gears are not inserted in the correct order, or if the shafts are not rotated appropriately during assembly,
it may be impossible to mesh the teeth properly due to interference from the large teeth profiles.
The robot must therefore coordinate the rotation of the shafts and the insertion of the gears to ensure that all teeth are able to engage correctly.

\noindent{\hspace*{\fill}\sphinxincludegraphics[width=400\sphinxpxdimen]{{gear_assembly_taskboard_three_gears}.png}\hspace*{\fill}}


\subparagraph{Test Objective:}
\label{\detokenize{robotic_instructions_gear_tasks:id5}}
\sphinxAtStartPar
This task evaluates the robot’s ability to recognize and align keyed components,
coordinate the placement of multiple gears, and ensure proper meshing among all gears.


\subparagraph{Success Criteria:}
\label{\detokenize{robotic_instructions_gear_tasks:id6}}\begin{itemize}
\item {} 
\sphinxAtStartPar
All three gears are mounted onto their respective shafts with correct alignment.

\item {} 
\sphinxAtStartPar
The grooves of the outer gears are aligned with the shaft keys.

\item {} 
\sphinxAtStartPar
The teeth of all gears are properly engaged, and the gears are fully seated on the assembly component.

\item {} 
\sphinxAtStartPar
The underside of each gear is in contact with the component surface, and all gears remain stationary after placement.

\end{itemize}

\sphinxstepscope


\subsubsection{Tasks involving Screws and Nuts}
\label{\detokenize{robotic_instructions_screws_and_nuts:tasks-involving-screws-and-nuts}}\label{\detokenize{robotic_instructions_screws_and_nuts::doc}}
\sphinxAtStartPar
Screws and nuts are fundamental fasteners used to join components in industrial assemblies.
In robotics, handling these parts requires precise manipulation, alignment, and force control to ensure reliable and repeatable assembly.
The following tasks evaluate the robot’s ability to autonomously perform screw and nut assembly operations under realistic constraints,
reflecting challenges encountered in automated industrial environments.
A task board with all three screw and nut assembly tasks completed successfully is shown below:

\noindent{\hspace*{\fill}\sphinxincludegraphics[width=400\sphinxpxdimen]{{screw_nut_assembly_taskboard_finished}.png}\hspace*{\fill}}


\paragraph{Task 1: Vertical Screw Insertion}
\label{\detokenize{robotic_instructions_screws_and_nuts:task-1-vertical-screw-insertion}}

\subparagraph{Description:}
\label{\detokenize{robotic_instructions_screws_and_nuts:description}}
\sphinxAtStartPar
A 20mm M8 screw is placed on a flat surface next to the Task Board.
The robot must pick up the screw, align it with a threaded insert on the Task Board, and insert it vertically.
The robot must ensure the screw is properly oriented and fully inserted without cross\sphinxhyphen{}threading or applying excessive force.

\noindent{\hspace*{\fill}\sphinxincludegraphics[width=400\sphinxpxdimen]{{screw_nut_assembly_taskboard_vertical_screw_insertion}.png}\hspace*{\fill}}


\subparagraph{Test Objective:}
\label{\detokenize{robotic_instructions_screws_and_nuts:test-objective}}
\sphinxAtStartPar
This task evaluates the robot’s ability to autonomously grasp, align, and insert a screw into a threaded insert, demonstrating  manipulation
and force control.


\subparagraph{Success Criteria:}
\label{\detokenize{robotic_instructions_screws_and_nuts:success-criteria}}\begin{itemize}
\item {} 
\sphinxAtStartPar
The screw is fully inserted into the thread insert.

\item {} 
\sphinxAtStartPar
The screw cannot be further tightened without excessive force.

\item {} 
\sphinxAtStartPar
The screw remains stationary after insertion.

\end{itemize}


\paragraph{Task 2: Nut Assembly with Limited Accessibility}
\label{\detokenize{robotic_instructions_screws_and_nuts:task-2-nut-assembly-with-limited-accessibility}}

\subparagraph{Description:}
\label{\detokenize{robotic_instructions_screws_and_nuts:id1}}
\sphinxAtStartPar
An M8 nut is placed on a flat surface next to the Task Board.
The robot must pick up the nut and insert it onto a screw protruding from the side of an assembly component.
Due to an overhang, the last 5mm of threading is especially difficult to access, requiring the robot to regrasp and reposition the nut during assembly.

\noindent{\hspace*{\fill}\sphinxincludegraphics[width=400\sphinxpxdimen]{{screw_nut_assembly_taskboard_nut_assembly}.png}\hspace*{\fill}}


\subparagraph{Test Objective:}
\label{\detokenize{robotic_instructions_screws_and_nuts:id2}}
\sphinxAtStartPar
This task evaluates the robot’s ability to handle nuts in constrained spaces, including regrasping and adapting its approach to overcome limited accessibility.


\subparagraph{Success Criteria:}
\label{\detokenize{robotic_instructions_screws_and_nuts:id3}}\begin{itemize}
\item {} 
\sphinxAtStartPar
The nut is fully threaded onto the screw until it cannot be tightened further by the robot.

\item {} 
\sphinxAtStartPar
The nut remains stationary after assembly.

\end{itemize}


\paragraph{Task 3: Multi\sphinxhyphen{}Screw Assembly with Alignment Constraints}
\label{\detokenize{robotic_instructions_screws_and_nuts:task-3-multi-screw-assembly-with-alignment-constraints}}

\subparagraph{Description:}
\label{\detokenize{robotic_instructions_screws_and_nuts:id4}}
\sphinxAtStartPar
Four 20mm M8 screws are placed on a flat surface next to the Task Board.
The robot must pick up each screw, align it with the corresponding holes in two housing parts, and insert them sequentially.
A key challenge in this task is avoiding canting—a phenomenon where tightening a single screw fully causes the parts to tilt or misalign due to uneven force distribution.
Canting is highly dependent on the geometry and tolerances of the parts being assembled.
To prevent this, the robot must use a sequential approach, partially tightening each screw in turn and only fully tightening them once all screws are in place, ensuring the two housing parts remain parallel with a maximum alignment error of 1°.

\noindent{\hspace*{\fill}\sphinxincludegraphics[width=400\sphinxpxdimen]{{screw_nut_assembly_taskboard_multi_screw_assembly}.png}\hspace*{\fill}}


\subparagraph{Test Objective:}
\label{\detokenize{robotic_instructions_screws_and_nuts:id5}}
\sphinxAtStartPar
This task evaluates the robot’s ability to perform multi\sphinxhyphen{}point screw assembly while maintaining precise alignment between components,
and deal with the challenges of canting during assembly.


\subparagraph{Success Criteria:}
\label{\detokenize{robotic_instructions_screws_and_nuts:id6}}\begin{itemize}
\item {} 
\sphinxAtStartPar
All four screws are fully inserted and tightened.

\item {} 
\sphinxAtStartPar
The two housing parts are screwed together and aligned parallel within 1°.

\end{itemize}

\sphinxstepscope


\subsubsection{Tasks with Elastic Deformation}
\label{\detokenize{robotic_instructions_elastic_deformation:tasks-with-elastic-deformation}}\label{\detokenize{robotic_instructions_elastic_deformation::doc}}
\sphinxAtStartPar
This section describes tasks that involve elastic deformation of components, requiring the robot to handle and manipulate parts that must be compressed or expanded during assembly.
In contrast to limp objects such as cables or hoses, these tasks involve components that snap back to their original shape after deformation, such as snap hooks, springs, and circlips.
A task board with all three tasks completed successfully is shown below.

\noindent{\hspace*{\fill}\sphinxincludegraphics[width=400\sphinxpxdimen]{{elastic_deformation_taskboard_finished}.png}\hspace*{\fill}}


\paragraph{Task 1: Snap Hook}
\label{\detokenize{robotic_instructions_elastic_deformation:task-1-snap-hook}}

\subparagraph{Description:}
\label{\detokenize{robotic_instructions_elastic_deformation:description}}
\sphinxAtStartPar
A snap hook must be mounted into a designated assembly component on the task board.
The snap hook is designed to elastically deform during insertion,
requiring it to be pushed into place so that its hooks compress and then expand to lock into the assembly.
The task challenges the robot to handle components that require elastic deformation, which introduces compliance and force control considerations.

\sphinxAtStartPar
The main difficulty lies in the need to apply enough force to deform the snap hook without damaging it,
and to ensure that the hook is fully seated and locked in place. The robot must also recognize when the snap hook is properly retained,
as the component cannot be removed without compressing the hooks again.

\noindent{\hspace*{\fill}\sphinxincludegraphics[width=400\sphinxpxdimen]{{snap_hook_placement_finished}.png}\hspace*{\fill}}


\subparagraph{Test Objective:}
\label{\detokenize{robotic_instructions_elastic_deformation:test-objective}}
\sphinxAtStartPar
This task evaluates the robot’s ability to manipulate and assemble components that require elastic deformation, and to verify secure placement.


\subparagraph{Success Criteria:}
\label{\detokenize{robotic_instructions_elastic_deformation:success-criteria}}\begin{itemize}
\item {} 
\sphinxAtStartPar
The snap hook is securely mounted inside the assembly component.

\item {} 
\sphinxAtStartPar
The hook cannot be removed without compressing the hooks.

\item {} 
\sphinxAtStartPar
The snap hook is fully seated and locked in place.

\end{itemize}


\paragraph{Task 2: Spring}
\label{\detokenize{robotic_instructions_elastic_deformation:task-2-spring}}

\subparagraph{Description:}
\label{\detokenize{robotic_instructions_elastic_deformation:id1}}
\sphinxAtStartPar
A spring must be compressed and inserted into two circular holders on the assembly component. The relaxed length of the spring is longer than the available space, so the spring must be compressed during placement. This task introduces the challenge of manipulating compliant objects that resist deformation and may slip or move unpredictably when compressed.

\sphinxAtStartPar
The difficulty for the robot is to maintain control of the spring while compressing it and positioning it accurately within the holders.
The robot must also ensure that the spring remains in place after release, as improper placement can cause the spring to pop out or not be fully seated.

\noindent{\hspace*{\fill}\sphinxincludegraphics[width=400\sphinxpxdimen]{{spring_placement_finished}.png}\hspace*{\fill}}


\subparagraph{Test Objective:}
\label{\detokenize{robotic_instructions_elastic_deformation:id2}}
\sphinxAtStartPar
This task assesses the robot’s ability to handle and assemble compliant, deformable objects that require forceful manipulation and precise placement.


\subparagraph{Success Criteria:}
\label{\detokenize{robotic_instructions_elastic_deformation:id3}}\begin{itemize}
\item {} 
\sphinxAtStartPar
The spring is fully positioned within the two circular holders.

\item {} 
\sphinxAtStartPar
The spring remains in place after release.

\item {} 
\sphinxAtStartPar
The spring is not protruding or misaligned.

\end{itemize}


\paragraph{Task 3: Circlip Assembly}
\label{\detokenize{robotic_instructions_elastic_deformation:task-3-circlip-assembly}}

\subparagraph{Description:}
\label{\detokenize{robotic_instructions_elastic_deformation:id4}}
\sphinxAtStartPar
A circlip, also known as a retaining ring or snap ring, is a type of fastener that fits into a machined groove on a shaft or inside a bore to secure components in place.
Circlips are typically made of spring steel and are designed to be elastically deformed during installation or removal.
They provide a removable shoulder that prevents lateral movement of parts, such as bearings or gears, along a shaft.

\sphinxAtStartPar
In this task, a circlip must be expanded and placed into a retaining ring groove on a shaft of the assembly component.
The circlip is a flexible ring that must be elastically deformed (expanded) to fit over the shaft, and then released so it contracts into the groove.
This task highlights the challenge of manipulating small, flexible parts that require precise deformation and alignment.

\sphinxAtStartPar
The main difficulty is expanding the circlip enough to clear the shaft without overstretching or dropping it, and then aligning it accurately with the groove before releasing.
The robot must also verify that the circlip is fully seated in the groove, as improper placement can result in incomplete assembly.


\subparagraph{Test Objective:}
\label{\detokenize{robotic_instructions_elastic_deformation:id5}}
\sphinxAtStartPar
This task evaluates the robot’s ability to manipulate and assemble small, flexible components that require controlled deformation and precise placement.


\subparagraph{Success Criteria:}
\label{\detokenize{robotic_instructions_elastic_deformation:id6}}\begin{itemize}
\item {} 
\sphinxAtStartPar
The circlip is correctly placed inside the groove of the assembly component.

\item {} 
\sphinxAtStartPar
The circlip is fully seated and retained in the groove.

\item {} 
\sphinxAtStartPar
The circlip does not fall off or sit above the groove after placement.

\end{itemize}


\paragraph{Task 3.1: Circlip Removal}
\label{\detokenize{robotic_instructions_elastic_deformation:task-3-1-circlip-removal}}

\subparagraph{Description:}
\label{\detokenize{robotic_instructions_elastic_deformation:id7}}
\sphinxAtStartPar
This task is a variation of Task 3, where the robot must remove a circlip from a retaining ring groove on a shaft of the assembly component.
The circlip is already installed, and the robot must expand it to clear the shaft and then remove it without damaging the circlip or the assembly component.
Difficulties lie not only in controlling the expansion of the circlip but also even just grasping the holes of the circlip without slipping.


\subparagraph{Test Objective:}
\label{\detokenize{robotic_instructions_elastic_deformation:id8}}
\sphinxAtStartPar
This task assesses the robot’s ability to handle and manipulate small, flexible components that require controlled deformation for removal.


\subparagraph{Success Criteria:}
\label{\detokenize{robotic_instructions_elastic_deformation:id9}}\begin{itemize}
\item {} 
\sphinxAtStartPar
The circlip is successfully removed from the groove without damage.

\item {} 
\sphinxAtStartPar
The circlip is placed in a designated area after removal.

\end{itemize}



\renewcommand{\indexname}{Index}
\printindex
\end{document}